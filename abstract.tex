Ultra/Luminous infrared galaxies (U/LIRGs) are some of the most amazing systems in the local universe exhibiting extreme star formation triggered by mergers. Since molecular gas is the fuel for star formation, studying the warm, dense gas associated with star formation is important in understanding the processes and timescales controlling star formation in mergers. We have used high resolution ($\sim$2.3") observations of the local LIRG Arp 299 (D= 44Mpc) to map out the physical properties of the molecular gas. The molecular lines $^{12}$CO J=3-2, $^{12}$CO J=2-1 and $^{13}$CO J=2-1 were observed with the Submillimeter Array and the short spacings of the $^{12}$CO J=3-2 and J=2-1 observations have been recovered using James Clerk Maxwell Telescope single dish observations. We use the radiative transfer code RADEX to measure the physical properties such as density and temperature of the different regions in this system. The RADEX solutions of the two galaxy nuclei, IC 694 and NGC 3690, show two gas components: a warm  moderately dense gas with $T_{kin}$ $\sim$ 30-500 K (up to 1000 K for NGC 3690) and $n$(H$_{2}$) $\sim$ 0.3 - 3 $\times$ 10$^{3}$ cm$^{-3}$ and a cold dense gas with $T_{kin}$ $\sim$ 10-30 K and $n$(H$_{2}$) $>$ 3 $\times$ 10$^{3}$ cm$^{-3}$. The overlap region is shown to have a well-constrained solution with $T_{kin}$ $\sim$ 10-30 K and $n$(H$_{2}$) $\sim$ 3-30 $\times$ 10$^{3}$ cm$^{-3}$. We estimate the gas masses and star formation rates of each region in order to derive molecular gas depletion times. The depletion time of each region is found to be about 2 orders of magnitude lower than that of normal spiral galaxies. This can be probably explained by a higher fraction of dense gas in Arp 299 than in normal disk galaxies. 
