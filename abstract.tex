This thesis presents new 450$\mu$m and 850$\mu$m observations of NGC3627 taken with the new SCUBA-2 with the main goal of trying to better understand the properties of gas and dust in the ISM of NGC3627.  We determined properties of the cold component of NGC3627's SED using dust models given by the Planck Collaboration, Li and Draine, and allowing the emissivity index to be treated as a free parameter.  Fitting the SED required the use of 100$\mu$m, 160$\mu$m, 250$\mu$m, 350$\mu$m, and 500$\mu$m data from the KINGFISH survey.  Each of the KINGFISH observations have passed through an exteded emission filter in order to match the SCUBA-2 observations.  The best fit temperatures and emissivity indicies were within agreement of the results found by in recent studies, but our fitted masses were below what the studies reported due to differences in the fitted temperature and observed fluxes.

After the properties of the dust emission were calculated, we implemented a method to determine the amount of molecular hydrogen present in NGC3627.  The method we used involves finding a CO-to-H$_2$ conversion factor that minimizes the scatter present in dust-to-gas mass ratio. We used CO J=2-1 from the HERACLES survey and CO J=1-0 from the Nobeyama 45-m telescope to act as our molecular tracer, and HI observations of NGC3627 from the THINGS survey.  The results from minimizing the dust-to-gas ratio scatter give low $\alpha_{CO}$ values that are normally associated with U/LIRGs.  The low $\alpha_{CO}$ values can be attributed to the treatment of the error associated with reported $\alpha_{CO}$.  The uncertainties for $\alpha_{CO}$ reported in this thesis are a minimum estimate, and if the error associated with $\alpha_{CO}$ is large enough, then the best fit $\alpha_{CO}$ values can be considered as a lower threshold for the system.
