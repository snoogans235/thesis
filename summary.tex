\chapter{Summary}

We have shown two new continuum maps of NGC3627 at 450$\mu$m and 850$\mu$m from SCUBA-2 and analyzed them with the goal of determining a dust-to-gas mass ratio, a suitable CO-to-H$_2$ conversion factor, and the amount of molecular hydrogen present, both in the galaxy as a whole and in 5 individual regions picked based upon the morphological features of the galaxy.  In order to produce our results we utilized data from the KINGFISH survey \citep{kennicutt2011} and the NGLS \citep{wilson2012} to calculate a dust mass using SED fitting on a pixel by pixel basis and also for the total flux of each region.  The dust-to-gas ratio, $\alpha_{CO}$, and molecular gas surface density were determined using a minimization of the dust-to-gas ratio scatter in the galaxy using data from the Nobeyama 45-m CO J=1-0 survey \citep{kuno2007}, HERACLES \citep{reuter1996}, and THINGS \citep{walter2008}.  The important results from our analysis are as follows:

\begin{enumerate}

\item{We have created high quality images of NGC3627 using the Starlink program MAKEMAP by incorporating masks in the AST and FLT portions of MAKEMAP and adding a high-pass filter of 175$\arcsec$ in the FLT portion of MAKEMAP in order to reduce the noise in the final image products.}

\item{Using results from the KINGFISH survey \citep{kennicutt2011}, we were able to show an excess emission was present in the SCUBA-2 450$\mu$m observations that was more likely due to calibration or mapping issues rather than a physical process in the ISM of NGC3627.}

\item{The results of the SED analysis were tested using two competing dust models calculated by the Planck observation group \citep{planckxxv2011} and Li and Draine \citep{li2001}.  A third model was tested using the opacity values from the Planck model with an emissivity index that was allowed to vary.  All three models were applied to the 5 regions of NGC3627 and the galaxy as a whole.  The results are shown in Tables \ref{tab:beta_1}, \ref{tab:beta_2}, \ref{tab:beta_f}, and \ref{tab:region_vals}.  The calculated temperatures for the entire galaxy are 23$\pm$2 K, 24$\pm$2 K, 22$\pm$1 K, and 22$\pm$2 K for the Planck model, Li and Draine model, free emissivity index, and regional flux, respectively.  The calculated masses using the Planck opacity are 52$\pm$23 $\times10^5$ M$_\odot$, 73$\pm$31 $\times10^5$ M$_\odot$, and 75$\pm$28 $\times10^5$ M$_\odot$ for the Planck model, free emissivity index, and regional flux, respectively.  The mass returned for the Li and Draine model is 230$\pm$103 $\times10^5$ M$_\odot$.  When the emissivity index was allowed to vary the returned values were 2.2$\pm$0.2 and 2.3$\pm$0.3 for the pixel by pixel fits and regional flux fits, respectively.  The results from the fitting agreed with previous work by \cite{galametz2012}.}

\item{There was no observed excess emission at 850$\mu$m in our fits, in contrast to the results of \cite{galametz2014} using 870$\mu$m maps from LABOCA.  The difference in the two results can be attributed to how they handled the molecular contribution at 870$\mu$m and whether the reduction process they used for the 870$\mu$m emission preserved any extended features suggesting the excess emission can be attributed to extended emission.}

\item{Using the Planck and the Li and Draine dust models to determine the dust-to-gas ratio showed that the Li and Draine model yielded higher dust-to-gas ratios than the Planck model.  This is due to the Li and Draine model having a higher dust surface density.}

\item{Using either the CO J=1-0 line or scaling the CO J=2-1 line using a 2-1/1-0 ratio does not affect the dust-to-gas ratio, but does change the CO-to-H$_2$ conversion factor.  Using the scaled CO J=2-1 emission results in higher CO-to-H$_2$ conversion factors, perhaps because the CO J=1-0 emission may trace more diffuse gas \citep{wilson1990} where atomic hydrogen will be more dominant than its molecular counterpart.}

\item{The gas data with the extended emission removed yielded $\alpha_{CO}$ values typical of U/LIRGs.  This was likely due to the significant portion of the HI emission that is made up of extended emission.  This extended HI emission was removed by the filtering process, which greatly decreased the HI surface density.  When using unfiltered gas data, we recovered conversion factors and dust-to-gas ratios similar to the results of \cite{sandstrom2013} using the same H$_2$ tracer.}

\end{enumerate}

Future work on NGC3627 and the NGLS SCUBA-2 catalogue would include:

\begin{enumerate}

\item{Investigating the excess emission in the 450$\mu$m image to determine its source.  If the source of this excess 450$\mu$m emission could be found, the resolution of our analysis could be improved and higher quality SED fitting could take place.}

\item{Using a bootstrap or similar error estimation method in order to obtain more reliable uncertainties on $\alpha_{CO}$ to gain more reliable errors on the dust-to-gas ratio and to determine the quality of the fits.}

\item{Implementing a MCMC method to determine $\alpha_{CO}$ values for each pixel would help to decrease the overall scatter of the dust-to-gas ratio and provide dust-to-gas ratio, $\alpha_{CO}$, and H$_2$ surface density maps to go with the SED fitting maps.  These maps could be used to determine correlations in $\alpha_{CO}$ with parameters returned from the SED fitting.}

\item{Implementing the SED and dust-to-gas ratio analysis for other targets in the NGLS catalogue would allow for a large scale statistical survey over the properties of the dust in the galaxies in local Universe.}

\end{enumerate}