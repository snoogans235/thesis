\chapter{Future Work}

This project can be improved by obtaining another $^{13}$CO transition line observation in order to better constrain the physical conditions. Another possibility to constrain the physical conditions is to obtain a dense gas tracer such as HCO+ or HCN; however adding these lines into our RADEX models will introduce more uncertainties because we are assuming another H$_{2}$ abundance ratio. The analysis can further be improved by recovering the short spacings of the $^{12}$CO J=1-0 and $^{13}$CO J=2-1 maps. We have proposed to observe Arp 299 in $^{13}$CO J=2-1 using the JCMT but the observations are currently not completed. Another option is to analyze the $Herschel$ FTS spectra of each region. The analysis will be difficult because the beam is large and will be contaminated with emission from other regions of Arp 299. 

Wilson et al. (2008) has a sample of 14 U/LIRGs observed with the SMA probing similar spatial scales as Arp 299. The other 13 U/LIRG observations have not been analyzed in great detail like Arp 299. The same procedure in this thesis can be used to obtain constrains on the physical conditions for each system. Once we have these constrains, we can correlate them to star formation properties and then compare each galaxy. We can also try to correlate the star formation properties to the merger stage. To expand this sample, we will propose to observe several more U/LIRGs in the south with the next generation observatory ALMA.  With the resolution and sensitivity of ALMA, we can probe regions on smaller scales and in some cases individual giant molecular clouds. Unfortunately, we cannot observe Arp 299 with ALMA. 