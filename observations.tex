\chapter{Observations and Data Preparation}\label{observations}

The Submillimetre Common-User Bolometer Array 2 (SCUBA-2) was designed to decrease the observing time of its predecessor SCUBA.  The decrease in observing time was obtained by increasing the number of pixels from 91 and 37 bolometers to over 10,000 pixels.  The camera is cooled to 4K and observes the sky in two wavelengths, 450$\mu$m and 850$\mu$m \citet{holland2013}.

The first science images were taken of NGC3627 in order to determine appropriate scan methods.  Observations were taken in grade 3 weather ($0.08<\tau<0.12$).  In all 'X' observations were taken of NGC3627, and 'Y' were used in the reduction.


--------

It might be nice to have one large table with all of the rms values, beams used for convolution, and beam sizes for each of the wavebands observed

\section{SCUBA-2}
\begin{itemize}
   \item What is SCUBA-2 and what does it look at and why is it an important/how is it relevant among today's instrumentation
   \item Discuss the nature of the observations and their reduction, things like weather grade, number of images used.  Technical details.
   \item maybe include figure of emission window for scuba-2
\end{itemize}

\subsection{Image Creation}
\begin{itemize}
   \item Use SMURF MAKEMAP command and give brief description of overall process.  Include $img_gen.sh$ in an appendix?
   \item Implement maps in the ast and flt portion of makemap.  Explain what these do for the image overall (should reduce noise and help to flatten background)
   \item use flt.filtering to remove any large scale features.  Helped with smoothing background noises.
   \item apply FCF values that are determined from calibrations from Uranus
   \item 850 gridded to 8sq'' and 450 gridded to 4sq'' pixels.
   \item for 850$\mu$m needed to remove $CO_{j=3-2}$ emission from continuum band.  Reference the drabek paper and a brief outline of what she did.  QUOTE THE NUMBER USED TO CONVERT UNITS.
\end{itemize}

\subsection{Image Properties of 450$\mu$m and 850$\mu$m}
\begin{itemize}
   \item show calibration data?
   \item beam shape of 450 and 850.  Discuss fitting methods for both single and double gaussian shapes also give rms values of noise before any convolution.
   \item show beams in color image w/contour and plot across the middle
\end{itemize}

\section{Supporting Images}

\begin{itemize}
  \item Discuss what you need other images for.  Herschel for SED fits.  THINGS, KUNO, and Heracles for dust-to-gas ratio.
  \item Images needed to be run through makemap as a fake source in order to remove any small scale structure from original images.  I should have several figures showing the amount of flux removed from each of the images.
  \item Reference the Kingfish Survey, Kuno et al., Heracles survey and THINGS survey for their sources.  <--Leave this as a paragraph on it's own or introduce smaller subsections?
  
\end{itemize}

