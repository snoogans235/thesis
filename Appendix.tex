\chapter{Miriad to CASA Comparison}\label{miriadtocasa}

CASA (Common Astronomy Software Applications) is a data analysis software package that is currently being developed for the next generation of radio telescopes, specifically ALMA and the EVLA \footnote{http://casaguides.nrao.edu/index.php?title=What\_is\_CASA$\%$3F}. The package can process both single dish and interferometric data. It is worthwhile to compare this new package to an older, more familiar package such as Miriad. We shall use the feathering technique of combining the JCMT single dish data with the interferometric SMA data of each package to compare the two.

\section{Miriad}

The JCMT single dish data was binned to 20 km/s (optical definition) starting at -389.9 km/s using the JCMT software, $Starlink$. The data cube was converted to a fits file, imported into Miriad and was prepared for combination with the SMA map. The intrinsic unit of the JCMT data is antenna temperature (T$_{A}$) and was converted to main beam temperature (T$_{mb}$) using equation \ref{antennatomb} since the emission fills the beam of the JCMT. The SMA data has units of Jy/beam, so we converted the JCMT data to the unit of Jy/beam using equation \ref{KtoJybeam} using $\eta_{mb}$ =0.6 for $\nu$ =345GHz. The header was modified to include the beam information and rest frequency. The SMA data uses the radio definition of velocity, so the JCMT data was converted to this definition using the \verb=velsw= task. The channels need to increase in velocity so the \verb=reorder= task is used to rearrange the data cube. The following is the script used to prepare the JCMT map.

\begin{verbatim}
fits in=Arp299co32.sub20.JCMT.K.fits op=xyin out=Arp299co32.JCMTsub20.TAmiriad

##Convert Ta* to Tmb by dividing by eta_mb =0.6 for JCMT at CO(3-2)
maths exp='Arp299co32.JCMTsub20.TAmiriad/0.6' 
	out=Arp299co32.JCMTsub20.Tmbmiriad

####Convert from K to Jy/BEAM using 2.786x10^7/(Area(cm^2)*eta_mb) = 26.28 
####Area = pi*(750cm)^2
maths exp='Arp299co32.JCMTsub20.Tmbmiriad*26.28' 
	out=Arp299co32.JCMTsub20.JyBEAMmiriad

###Fix the header information 
puthd in=Arp299co32.JCMTsub20.JyBEAMmiriad/bunit value=JY/BEAM
puthd in=Arp299co32.JCMTsub20.JyBEAMmiriad/bmaj value=0.000067874
puthd in=Arp299co32.JCMTsub20.JyBEAMmiriad/bmin value=0.000067874
puthd in=Arp299co32.JCMTsub20.JyBEAMmiriad/bpa value=0
puthd in=Arp299co32.JCMTsub20.JyBEAMmiriad/btype value=Intensity type=ascii
puthd in=Arp299co32.JCMTsub20.JyBEAMmiriad/restfreq value=342.3 type=real
puthd in=Arp299co32.JCMTsub20.JyBEAMmiriad/ctype3 value=FELO-LSR

###Need to transform the optical velocity to the radio definition of velocity
velsw in=Arp299co32.JCMTsub20.JyBEAMmiriad axis=radio

#Need to reverse the order of the velocity axis 
reorder in=Arp299co32.JCMTsub20.JyBEAMmiriad mode=12-3 
	out=Arp299co32.JCMTsub20.JyBEAMmiriad.reordered

###Export the prepared JCMT data cube into a fits file
fits in=Arp299co32.JCMTsub20.JyBEAMmiriad.reordered op=xyout 
	out=Arp299co32.JCMTsub20.JyBEAM.fits
\end{verbatim}

For the SMA data, we first continuum subtracted the $uv$ data set using the task \verb=uvlin=. In order to use this task, we determined which channels are emission line free. In order to get a dirty mosaic map, we used the \verb=invert= task. The dirty mosaic map was binned to 20 km/s starting at -390.11 km/s and a robust weighting scheme was used to get a good balance of resolution and sensitivity. The \verb=mossdi= task was used to get a clean model and then the \verb=restor= task to convolve the clean model with the beam to obtain the clean mosaic map. We implemented a 2$\sigma$ cutoff to the clean model in order not to clean too deeply. The \verb=mossen= task is used to obtain a gain map in order to correct the clean map for the primary beam pattern. The JCMT map is regridded to match the SMA map and the two maps are combined using the \verb=immerge= task. The following is the procedure used to prepare the SMA map and combine it with the JCMT map:

\begin{verbatim}
uvlist vis=arp299f1_l.uv options=spectra
#window 2 starts at -760.818
#window 23 ends at 832.643
#velocity interval is 0.712
uvaver vis=arp299f1_l.uv  line=velocity,2239,-760.818,0.712,0.712 
       out=arp299f1_l.uv.vel
uvaver vis=arp299f2_l.uv  line=velocity,2239,-760.818,0.712,0.712 
       out=arp299f2_l.uv.vel

uvlin vis=arp299f1_l.uv.vel out=arp299f1_l.uv.nocont order=1 
      chans=1,600,1639,2239 offset=-20.1,1.3
uvlin vis=arp299f2_l.uv.vel out=arp299f2_l.uv.nocont order=1 
      chans=1,600,1639,2239 offset=-1.5,1.2

invert vis=arp299f1_l.uv.nocont,arp299f2_l.uv.nocont map=Arp299co32.im 
       beam=Arp299co32.beam options=double,systemp,mosaic imsize=512 
       cell=0.2arcsec robust=0.5 line=velocity,40,-390.1,20 sup=38
##Theoretical rms noise: 6.575E-02

#Check the created map
cgdisp in=Arp299co32.im device=/xs
#First 7 and last 4 channels noise. Use to determine rms

histo in=Arp299co32.im region="quarter(1,7)"
##Mean -3.603604E-05 Rms 3.53284E-02 Sum -1.94259E+01 (539070 points)
##Maximum value      1.346297E-01    at (172,341,6,1)
##Minimum value     -1.534902E-01    at (351,287,3,1)

histo in=Arp299co32.im region="quarter(36,40)"
##Mean -1.680603E-04 Rms    3.75209E-02 Sum  -6.47116E+01 (385050 points)
##Maximum value      1.757702E-01    at (301,337,37,1)
##Minimum value     -1.453050E-01    at (425,265,39,1)

#Take average as rms...3.64e-2
###Clean the image plane with a 2 sigma cutoff
mossdi map=Arp299co32.im beam=Arp299co32.beam out=Arp299co32.clean 
      region=quarter niters=2000 cutoff=0.0728
restor map=Arp299co32.im beam=Arp299co32.beam model=Arp299co32.clean 
      out=Arp299co32.map
#Using gaussian beam fwhm of    2.236 by    1.848 arcsec.
#Position angle:  -15.8 degrees.

###Correct for the primary beam
mossen in=Arp299co32.map gain=Arp299.map.gain sen=Arp299.map.sen
maths exp="Arp299co32.map/Arp299.map.gain" mask=Arp299.map.gain.gt.0.5 
     out=Arp299co32.mapg1

regrid in=Arp299.JCMT.JyBEAM.reorder.miriad 
     out=Arp299.JCMT.JyBEAM.regrid.miriad axes=1,2,3 tin=Arp299co32.mapg1

immerge in=Arp299co32.mapg1,Arp299.JCMT.JyBEAM.regrid.miriad 
     out=Arp299co32.feather.cube.miriad factor=1

moment in=Arp299co32.feather.cube.miriad region="quarter(8,35)" clip=0.0728 
      out=Arp299co32.feather.mom0

moment in=Arp299co32.mapg1 region="quarter(8,35)" clip=0.0728 
      out=Arp299co32.mapg1.mom0
\end{verbatim}

\begin{deluxetable}{cccccccc}
\tablecolumns{8}
\tablewidth{0pt}
\tablecaption{Miriad SMA and SMA+JCMT Comparison}
\label{miriadstats}
\tablehead{\colhead{} & \multicolumn{3}{c}{SMA} & \colhead{} & \multicolumn{3}{c}{SMA+JCMT} \\ \cline{2-4} \cline{6-8} \\ \colhead{Region} & \colhead{Maximum} & \colhead{Minimum} & \colhead{Flux} & \colhead{} & \colhead{Maximum} & \colhead{Minimum} & \colhead{Flux} \\ \colhead{} & \colhead{(Jy/beam)} & \colhead{(Jy/beam)} & \colhead{(Jy km s$^{-1}$)} & \colhead{} & \colhead{(Jy/beam)} & \colhead{(Jy/beam)} & \colhead{(Jy km s$^{-1}$)} }
\startdata
A & 857.97 & 1.47& 1350 & & 888.49 & 1.46 & 1640 \\
B & 303.44 & 1.46 & 711 & & 311.56 & 1.48 & 830 \\
C& 124.20 & 1.46 & 489 & & 132.03 & 1.46 & 723 \\
 \enddata
 \end{deluxetable}



\begin{figure}[h]
\centering
$\begin{array}{c}
\includegraphics[ clip,scale=0.7]{ThesisFigures/AppendixFigures/Miriad/Arp299co32mom01.png} \\
\includegraphics[ clip,scale=0.7]{ThesisFigures/AppendixFigures/Miriad/featherco32mom01.png} \\
\end{array}$
\caption[Integrated intensity maps created using Miriad for $^{12}$CO J=3-2]{Integrated intensity maps created using Miriad for $^{12}$CO J=3-2:(\textit{top}) SMA-only map, (\textit{bottom}) Feathered map. The contour levels are -10, -9,..., 9, 10, 15, 20, 25, 30, 35, 40, 45, 50 times 2$\sigma$ (=7.3 Jy/beam km/s). The dashed contours indicate missing flux. The white patches spread across the map indicate masked out pixels that are below 2$\sigma$. The black ellipse in the bottom left corner is the synthesized beam size. }
\label{miriadfeather}
\end{figure}

\section{CASA}

In CASA (v3.2.1), the $uv$ data set is Fourier transformed, cleaned and primary beam corrected within the one task \verb=clean=. We started by importing the SMA $uv$ data sets into the CASA format and concatenate the sets into one file. The continuum is subtracted from the data using the \verb=uvcontsub2= task, inputing the spectral windows that are line free (see Figure \ref{uvcont}) and assuming the continuum can be fit by a linear equation. The data is binned to 20 km/s starting at -390.1km/s to match the JCMT data cube. A robust weighting scheme was implemented as used in Miriad and a threshold of 2$\sigma$ in order not to clean too deeply. The following procedure was taken in CASA:

\begin{figure}[h]
\centering
%$\begin{array}{c}
\includegraphics[ clip,scale=0.7]{ThesisFigures/ObservationsFigures/AmpvsFreqco32.png} 
\includegraphics[ clip,scale=0.7]{ThesisFigures/ObservationsFigures/ampvsspw.png} 
%\end{array}$
\caption[Amplitude versus Frequency and Amplitude versus Spectral Window plots for $^{12}$CO J=3-2]{($top$): Amplitude versus Frequency plot for $^{12}$CO J=3-2. The CO line is prominent.; ($bottom$): Amplitude versus Spectral Window plot for $^{12}$CO J=3-2. The CO line is contained with spectral windows 8-15. }
\label{uvcont}
\end{figure}

\begin{verbatim}
os.system('mkdir arp299f1_ms_files')
myfiles=[]

for i in range(1,25):
        msfile="arp299f1_ms_file/arp299f1.spw"+str(i)+".ms"
        importuvfits(fitsfile="arp299f1_l.spwfits/arp299f1_1.spw"+str(i),
                     vis=msfile)
        myfiles.append(msfile)
	
os.system('mkdir arp299f2_ms_files')
myfiles2=[]

for i in range(1,25):
        msfile2="arp299f2_ms_file/arp299f2.spw"+str(i)+".ms"
        importuvfits(fitsfile="arp299f2_l.spwfits/arp299f2_1.spw"+str(i), 
                     vis=msfile2)
        myfiles2.append(msfile2)
	
concat(vis=myfiles,concatvis="arp299f1.uv.ms",timesort=True)
concat(vis=myfiles2,concatvis="arp299f2.uv.ms",timesort=True)
concat(vis=["arp299f1.uv.ms","arp299f2.uv.ms"],concatvis="Arp299co32.uv.ms",
     timesort=True)

plotms(vis="Arp299co32.uv.ms",xaxis="freq",yaxis="amp",ydatacolumn="data",
     selectdata=True,coloraxis="spw",spw="0~23",averagedata=True,
     avgtime="99999",avgscan=T)

uvcontsub2(vis="Arp299co32.uv.ms",want_cont=T,combine='spw',
     fitspw='1~7:5~122,16~22:5~122',spw='',fitorder=1)

clean(vis="Arp299co32.uv.ms.contsub",imagename="Arp299co32.20",
     mode="velocity",start="-390.1kms",width="20kms",nchan=40,
     niter=1000,threshold="10mJy",ftmachine="ft",imagermode="mosaic",
     interactive=T,interpolation="nearest",imsize=512,cell="0.2arcsec",
     pbcor=T,weighting="briggs",robust=0.5)
#Beam used in restoration: 2.19338 by 1.87133 (arcsec) at pa -18.0679 (deg) 
\end{verbatim}

One issue with combining the JCMT and SMA maps in CASA is that the JCMT map does not have a fourth axis, the Stokes axis. CASA does not like to combine maps without the same number of axes, therefore we need to remove the Stokes axis from the SMA map or add one to the JCMT map. One way to remove the Stokes axis from the SMA map is to export it into a fits file with the parameter \verb=dropstokes= set to \verb=True= and then import it back into CASA. The hope is that this ``issue" will be fixed in upcoming versions. Another problem with CASA is that it doesn't import fits files from Starlink properly. We need to import and export the fits file in Miriad first. This however does not retain the beam information of the JCMT map when imported into CASA. We need to add this information back into the header. The two maps can be combined using the \verb=feather= task without the need to regrid the JCMT map. After the combination, a integrated intensity map is created using the channels with line emission and pixels greater than 2$\sigma$. The following is the full procedure taken in CASA:

\begin{verbatim}
exportfits(imagename="Arp299co32.20.image",fitsimage="SMA.fits",dropstokes=T)
importfits(fitsimage="SMA.fits",imagename="Arp299co32.20.image.nostokes")


imhead(imagename="Arp299.JCMT.JyBEAM.image",mode="put",hdkey="beammajor",
     hdvalue="14arcsec")
imhead(imagename="Arp299.JCMT.JyBEAM.image",mode="put",hdkey="beamminor",
     hdvalue="14arcsec")
imhead(imagename="Arp299.JCMT.JyBEAM.image",mode="put",hdkey="beampa",
     hdvalue="0deg")

feather(imagename="Arp299.feather.cube",highres="Arp299co32.20.image.nostokes",
     lowres="Arp299.JCMT.JyBEAM.image")

imstat(imagename="Arp299.feather.cube",chans="0~7",box="111,204,274,319")
 #'rms': array([ 0.03409981]),

immoments imagename="Arp299.feather.cube",outfile="Arp299.feather.mom0",
     moments=[0],axis="spectral",chans="7~35",includepix=[0.068,10000] 
\end{verbatim}

\begin{deluxetable}{cccccccc}
\tablecolumns{8}
\tablewidth{0pt}
\tablecaption{CASA SMA and SMA+JCMT Comparison}
\label{casatable}
\tablehead{\colhead{} & \multicolumn{3}{c}{SMA} & \colhead{} & \multicolumn{3}{c}{SMA+JCMT} \\ \cline{2-4} \cline{6-8} \\ \colhead{Region} & \colhead{Maximum} & \colhead{Minimum} & \colhead{Flux} & \colhead{} & \colhead{Maximum} & \colhead{Minimum} & \colhead{Flux} \\ \colhead{} & \colhead{(Jy/beam)} & \colhead{(Jy/beam)} & \colhead{(Jy km s$^{-1}$)} & \colhead{} & \colhead{(Jy/beam)} & \colhead{(Jy/beam)} & \colhead{(Jy km s$^{-1}$)}}
\startdata
A & 872.56 & 1.36 & 1464  & & 892.32 & 1.38 & 1693  \\
B & 307.86 & 1.36 & 745 & & 306.63 &1.36 & 773 \\
C & 138.90 & 1.36 & 629 & & 138.95 & 1.36 & 775 \\
 \enddata
 \end{deluxetable}
 
 \begin{figure}[h]
\centering
$\begin{array}{c}
\includegraphics[trim=1cm 18cm 1cm 1cm, clip,scale=0.8]{ThesisFigures/AppendixFigures/CASA/CASASMAco32.pdf} \\
\includegraphics[trim=1cm 18cm 1cm 1cm, clip,scale=0.8]{ThesisFigures/AppendixFigures/CASA/CASAfeatherco32.pdf} \\
\end{array}$
\caption[ntegrated intensity maps created using CASA for $^{12}$CO J=3-2]{Integrated intensity maps created using CASA for $^{12}$CO J=3-2: (\textit{top}) SMA-only map, (\textit{bottom}) Feathered map. The contour levels are -10, -9,..., 9, 10, 15, 20, 25, 30, 35, 40, 45, 50 times 2$\sigma$ (=7.3 Jy/beam km/s). The dashed contours indicate missing flux. The white patches spread across the map indicate masked out pixels that are below 2$\sigma$. The white ellipse in the bottom left corner is the synthesized beam size.}
\label{casafeather}
\end{figure}
 
The main goal of combining single dish and interferometric data is to remove the negative bowls by adding the missing flux due to the short spacing problem. We see that the feathering technique does not recover all the missing flux (there still exists negative flux; only 67$\%$ of flux from JCMT recovered) but it does a fairly good job at removing the negative bowls near the main sources of emission in Arp 299 using both Miriad and CASA.  However, we do see differences in the two reduction packages. Comparing the peak values of each region in Arp 299 from Table \ref{miriadstats} and Table \ref{casatable}, we see that the values differ before and after combination. This may be because the robust weighting scheme may differ from each package. However, we do see that the rms values and the beam used in restoration are similar. Comparing Figure \ref{miriadfeather}(b) and \ref{casafeather}(b), the feathered maps, we do notice some differences. The region north-east of IC 694 is mostly recovered using CASA but with Miriad we still see some of major negative bowls remaining. This may indicate that the task in Miriad adds more flux around regions with positive flux while CASA attempts to disperse the flux uniformly in the map. 