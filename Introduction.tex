\chapter{Introduction}\label{intro}

\section{The Physical Conditions of Star Formation}

Star formation is one of the fundamental processes in astrophysics that effect not just stellar and planetary formation but also reaches into galaxy formation and evolution \citep{kennicutt2012}.  The study of star formation itself can be broken into several areas such as: the processes that trigger collapse and how the collapse behaves, dictating how and what type of stars can form from a given region or cloud, or how the inflow of new material can alter current star forming environments.  The different sub-branches of star formation are well sorted into a hierarchal schemes that range from Mpc scales seen in gas accretion from the intergalactic medium, ending with scales on the order $R_\odot$ or AU \citep{kennicutt2012}.  


Looking at how gmcs are formed and what role dust plays in the formation of molecules.

%Try to use the word obnubliates!


%\begin{itemize}
%  \item Components necessary in star formation and maybe look at how differing levels have an effect?
%    \begin{itemize}
%      \item This would be good to look at the physical conditions in molecular clouds.  Introduce the idea of how it is hard to observe H2 and we can use other tracers to see it.  The Evans paper is pretty good with this.
%      \item It might be good to talk about the processes in star formation and then look at what phases we 
%    \end{itemize}

%  \item Dust to gas ratio
%    \begin{itemize}
%      \item What the DGR can tell us about the sfr environment / it's importance this would include things like what different values can tell us
%      \item Components and how to calculate it.  Go into the need for $\alpha_{co}$
%      \item Leroy minimization technique and it's history and then new applications (sandstrom paper).
%    \end{itemize}

%  \item Spectral Energy Distribution of Dust
%    \begin{itemize}
%      \item What SED's can tell us and their comonents.  I believe Draine has a pretty kick ass paper on this.
%      \item Go into cold component and derive the modified black body.  This would be a good time to go over what $\kappa$ and $\beta$ are.
%      \item Some properties of cold dust emission pretty much talk about the 850 excess and go over the galametz paper that chris sent.
%    \end{itemize}

%  \item NGC3627 on a whole
%    \begin{itemize}
%      \item go into previous studies  There is the reuters paper, all of the kingfish stuff, accuarate distances with Ia SNe.  Look for information regarding the nucleus and an slight AGN.  Tidal tails with HI in leo triplet.
%    \end{itemize}
%\end{itemize}
