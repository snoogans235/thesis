\chapter{Introduction}\label{intro}

\section{The Physical Processes of Star Formation} %need to change this since it deals with formation of H2

Star formation is one of the fundamental processes in astrophysics that effects not just stellar and planetary formation but also dictates the behavior in galaxy formation and evolution \citep{kennicutt2012}.  The study of star formation itself can be broken into several areas such as: the processes that trigger collapse and how the collapse behaves, dictating how and what type of stars can form from a given region or cloud, or how the inflow of new material can alter current star forming environments.  These different sub-branches of star formation are conviently sorted into a hierarchal schemes that range from Mpc scales seen in gas accretion from the intergalactic medium, to properties of accretion disks on the a scales to the order $R_\odot$ or AU\citep{kennicutt2012}.  In this body of work we focus on the physical properties of the dust and abundance of gasses in the giant molecular cloud (GMC) phase of star formation prior to stellar collapse or fragmentation.

\subsection{GMC Formation}

After the intergalactic gas has accreted into a host galaxy, it will eventually condense and begin to form a GMC \citep{kennicutt2012}.  The condensation of this gas leads to the formation of molecular clouds.  The dominant process of GMC formation has divided between two camps; either a ``bottom-up'' or ``top-down'' formation scenario\citep{mckee2007}.  The bottom-up scenario consists of small clouds of cold HI coagulating to eventually form a GMC \citep{field1965, kwan1979}.  The major concern with the bottom-up scenario is if we include heating mechanisms in the cloud, coagulation will cease before the observed masses are reached \citep{mckee2007}.  In fact, the time scales required to form a cloud typical of what we observe, would take around $10^8$ years to accumulate a minimum of 10$^5$M$_\odot$, which is much greater than the expected GMC lifetime \citep{mckee2007}.  %try to find typical gmc lifetime 

The alternative formation scenario, top-down, postulates the GMC formation comes from instabilities in diffuse ISM causing the clouds to collapse from their surrounding medium \citep{mckee2007}.  Two main instabilities are thought to be responsible for the collapse.  The first type of instability is a Parker instability, which involves distortion of magnetic fields lines in the mid plane of the galaxy, and at these distortions gas will begin to accumulate \citep{parker1966, dobbs2013}.  The second instability responsible for collapse comes in the style of a more complex Jeans instability which is determined further by the amount of rotational shear present \citep{mckee2007}.  If a strong rotational shear is present, such as in the inter arm region of a spiral galaxy, then a process known as swing amplification will occur \citep{mckee2007}.  If no rotational shear is present, such as the inner regions of a galaxy or spiral arms, then the collapse can be attributed to a magneto-Jeans instability \citep{elmegreen1987,kim2001}.

\subsection{Molecular Hydrogen Formation}

Regardless of either collapse or coagulation, molecular hydrogen is being formed inside of the cloud.  The formation of molecular hydrogen can be summarized using either the two bodied reaction, three bodied reaction, formation using free a electron or proton, or surface formation \citep{krumholz2014}.  The two bodied formation scenario is the simplest reaction using two hydrogen atoms to produce molecular hydrogen, reaction \ref{eq:2bod}.

\begin{equation}\label{eq:2bod}
  \ce{H + H -> H_2}
\end{equation}

However the two bodied formation is not the major mechanism in the formation of H$_2$ due to the requirement of a forbidden photon that arises from the combination of hydrogen atoms in the ground state \citep{gould1963}.  If one of the hydrogen atoms is excited, the transition can occur and molecular hydrogen is formed, but the amount of excited hydrogen atoms in the temperature ranges typical of the cold and warm phases of the ISM are expected to be nearly nonexistent \citep{krumholz2014}.

The second formation scenario listed, the three bodied formation, involves three hydrogen atoms coming together to form molecular hydrogen with a spare hydrogen atom shown in reaction \ref{eq:3bod}.

\begin{equation}\label{eq:3bod}
  \ce{3H -> H_2 + H}
\end{equation}

The required density for the three bodied formation to occur is on the order of 10$^8$ cm$^{-3}$ \citep{palla1983,abel1997}, while the typical GMC density is on the order of 300 cm$^{-3}$.  The disparity between these densities eliminates any possibility of this begin the primary mechanism to form molecular hydrogen. 

An alternative to the two or three bodied reactions uses either an electron or proton to ionize the hydrogen forming either H$^-$ or H$_2^+$ \citep{krumholz2014}.  The chemical reaction involving an electron is shown in reaction \ref{eq:eform} and reaction utilizing a proton is shown in reaction \ref{eq:eform}
\begin{equation}\label{eq:eform}
  \begin{split}
    \ce{H + \it{e} -> H^- + \it{h\nu}} \\
    \ce{H^- + \it{e} -> H_2 + \it{e}}
  \end{split}
\end{equation}

\begin{equation}\label{eq:pform}
  \begin{split}
    \ce{H + H^+ -> H^+_2 + \it{h\nu}} \\
    \ce{H^+_2 + H -> H_2 + H^+}
  \end{split}
\end{equation}

The main limitation of the free electron/proton formation mechanism is an undersupply of free electrons and protons.  Typical Milky Way conditions show that in regions $>$1 cm$^{-3}$ have free electron and proton densities $<$10$^{-4}$ cm$^{-3}$ \citep{wolfire2003}.  Secondly, the H$^-$ ion in reaction \ref{eq:eform} is more likely to interact with a stray photon rather than an electron resulting in ion to return to the atomic state \citep{glover2003}.  While we observe singly ionized hydrogen regions, HII regions, only a small fraction of the region will successfully produce molecular hydrogen.

While the free electron/proton formation method is suspected to be the primary H$_2$ catalyst in the early universe \citep{herbst2005}, at low redshifts the formation of molecular hydrogen on the surface of dust grains is the dominant mechanism of formation \citep{krumholz2014}.  The surface formation of molecular hydrogen will occur when a hydrogen atom strikes a dust grain and successfully sticks to the grain.  The hydrogen then will then interact with another hydrogen atom to form H$_2$ and be ejected off of the dust particle after the reaction has occurred \citep{pirronello1997}.  The same process occurs on the dust grain as the two bodied formation, but  the dust grain will act as a medium to absorb the energy that would create the forbidden photon \citep{krumholz2014}.

%it would be a good idea to explain the requirements of the dust since not all dust can form H_2

With a dominant mechanism for molecular hydrogen formation, a reaction rate can be defined based on the cross section of the grain,$\Sigma_{gr}$, a striking probability dependent on the temperature of the observed dust, S(T), the probability of molecular hydrogen forming on the grain, $\upepsilon_{H_2}$, density of hydrogen nucleon, n$_H$, and the density of free hydrogen in the GMC, n$_H{_0}$ \citep{krumholz2014}.  Scaling this with the integrated collisional probability of a Maxwellian gas, a reaction rate is determined to be equation \ref{eq:reac_rate} where k is the Boltzmann constant, T is the temperature, and m$_H$ is the mass of the hydrogen atom.

\begin{equation}\label{eq:reac_rate}
  \frac{dn_{H_2}}{dt} = \frac{1}{2}\left(\frac{8k_bT}{\pi m_H}\right)^\frac{1}{2}\Sigma_{gr}S\left(T\right)\upepsilon_{H_2}n_{H}n_{H_0}
\end{equation}

Equation \ref{eq:reac_rate} is often simplified to equation \ref{eq:reac_rate_sm} by introducing a variable known as the formation rate, $\mathcal{R}_{gr}$, that is constrained using the column densities CI, CII, HI, H$_2$.  Typical reaction rates for the Milky Way have been found to be 3x10$^{-17}$cm$^3$s$^{-1}$ \citep{jura1975, gry2002, wolfire2008}.

\begin{equation}\label{eq:reac_rate_sm}
  \frac{dn_{H_2}}{dt} = \mathcal{R}_{gr}n_H n_{H_0}
\end{equation}

\subsection{Dissasociation of Moleuclar Hydrogen}

When the molecular hydrogen has formed, it is still susceptible to photodissasociation from far ultra-violet (FUV) photons.  The energy required for a single photon to break the bonds of molecular hydrogen is 14.5 eV \citep{krumholz2014}.  Conveniently, this is also enough energy to excite atomic hydrogen, so photodissasociation using a single photon is highly unlikely due to the abundance of HI in the ISM \citep{krumholz2014}.  However, a photon with an energy of 11-13.6 eV will not be able to excite atomic hydrogen, but will be able to excite the molecular hydrogen to its first and second excitation levels, the Lyman and Werner bands respectively.  The excited H$_2$ will eventually settle via photon emission to it's ground state with a finite probability of returning to two hydrogen atoms rather than maintaing it's molecular state \citep{krumholz2014}.

%need a little more in this use the draine 2011 book
A disassociation rate, $\zeta_{diss}$, can be obtained by scaling the total excitation rate, $\zeta_{exc}$, by the fraction of excited hydrogen molecules that will settle to an atomic ground state \citep{krumholz2014}.  The total excitation rate is found by summing each of individual excitations from the ground state e.g. $\zeta_{exc,0-1}$, $\zeta_{exc,0-2}$.  In the Milky Way's diffuse ISM, the interstellar radiation field is 6-9x10$^{-14}$ erg cm$^{-3}$ over the range of 6-13.6 eV resulting in $\zeta_{exc} \approx 3\times10^{-10}$s$^{-1}$ \citep{draine2011}.  The expected fraction of H$_2$ to disassociate is between 0.11 and 0.13 resulting in $\zeta_{diss}\approx4\times10^{-11}$s$^{-1}$\citep{draine2011}.
Equating the formation and disassociation rates then solving for a molecular hydrogen vs atomic hydrogen ratio is shown in equation \ref{eq:fd_eq1} \citep{krumholz2014}.  From equation \ref{eq:fd_eq1} we can see, in the diffuse medium, atomic hydrogen is far more abundant which is of no surprise.

\begin{equation}\label{eq:fd_eq1}
  \begin{split}
    \zeta_{diss} n_{H_2} & = \mathcal{R}n_{H_0}n_H \\
    \frac{n_{H_2}}{n_H} & = \frac{\mathcal{R}n_{H_0}}{\zeta_{diss}} \\
                        & = 8\times10^{-6}\left(\frac{4\times10^{-11}s^{-1}}{\zeta_{diss}}\right)\left(\frac{n_{H_0}}{10cm^{-3}}\right)
  \end{split}
\end{equation}

However, as the density of the gas increases via collapse or coagulation, the optical depth will also increase limiting the amount of FUV photons able to penetrate into the core of the cloud; a process known as shielding \citep{draine2011}.  This shielding will aid in the decrease of $\zeta_{diss}$ which will increase the value in equation \ref{eq:fd_eq1}.  The increase in molecular to atomic hydrogen ratio signifies the accumulation of molecular gas reservoirs.  The molecular gas build up will lead to fragmentation within the cloud, and will eventually lead to forming stars.  

\section{Determining H$_2$ Abundance}

%Was thinking something like:
%As shown in the above section, molecular hydrogen will be the dominant molecule in GMCS, so determining the quantity of H2 present is important for learning more about the star formation present.  Observing H2 however is not as easy as it sounds.  We know from \ref{sec_diss} that if the hydrogen is emitting a photon there is a finite chance that what we are seeing is H2 being debased into HI as well as leaving it to be too hot.  Secondly if we want to observe cold H2 via rotation, we cannot because of the lack of a dipole moment.  In order to really understand how much H2 is present we use tracers that we know will be present if H2 is present and then based on the amount of tracers we can make a determination of how much H2 is present.  This determiniation is done using a value known as the conversion factor , X_CO or \alpha_CO

%Try to use the word obnubliates!
