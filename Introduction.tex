\chapter{Introduction}\label{intro}

\section{The Physical Processes of Star Formation} %need to change this since it deals with formation of H2

Star formation is one of the fundamental processes in astrophysics that affects not just stellar and planetary formation but also dictates the behavior in galaxy formation and evolution \citep{kennicutt2012}.  The study of star formation itself can be broken into several areas such as: the processes that trigger collapse and how the collapse behaves, dictating how and what type of stars can form from a given region or cloud, or how the inflow of new material can alter current star forming environments.  These different sub-branches of star formation are conveniently sorted into hierarchal schemes that range from Mpc scales seen in gas accretion from the intergalactic medium, to properties of accretion disks on scales of the order $R_\odot$ or AU\citep{kennicutt2012}.  In this body of work we focus on the physical properties of the dust and abundance of gas in the giant molecular cloud (GMC) phase of star formation prior to stellar collapse or fragmentation.  However based on the resolution of our target galaxy, we will not be able to examine individual GMCs instead we will need to focus our work on giant molecular associations (GMAs).

\subsection{GMC Formation}

After the intergalactic gas has accreted into a host galaxy, it will eventually condense and begin to form a GMC \citep{kennicutt2012}.  The condensation of this gas leads to the formation of molecular clouds.  The dominant process of GMC formation is divided between two camps; either a ``bottom-up'' or ``top-down'' formation scenario \citep{mckee2007}.  The bottom-up scenario consists of small clouds of cold HI coagulating to eventually form a GMC \citep{field1965, kwan1979}.  The major concern with the bottom-up scenario is if we include heating mechanisms in the cloud, coagulation will cease before the observed masses are reached \citep{mckee2007}.  In fact, the time scales required to form a cloud typical of what we observe would take around $10^8$ years to accumulate a minimum of 10$^5$ M$_\odot$, which is much greater than the expected GMC lifetime \citep{mckee2007}.  %try to find typical gmc lifetime 

The alternative formation scenario, top-down, postulates the GMC formation comes from instabilities in the diffuse ISM causing the clouds to collapse from their surrounding medium \citep{mckee2007}.  Two main instabilities are thought to be responsible for the collapse.  The first type of instability is a Parker instability, which involves distortion of magnetic fields lines in the mid plane of the galaxy, and at these distortions gas will begin to accumulate \citep{parker1966, dobbs2013}.  The second instability responsible for collapse comes in the style of a more complex Jeans instability which is determined further by the amount of rotational shear present \citep{mckee2007}.  If a strong rotational shear is present, such as in the inter arm region of a spiral galaxy, then a process known as swing amplification will occur \citep{mckee2007}.  If no rotational shear is present, such as the inner regions of a galaxy or spiral arms, then the collapse can be attributed to a magneto-Jeans instability \citep{elmegreen1987,kim2001}.

\subsection{Molecular Hydrogen Formation}\label{h2form}

Regardless of either collapse or coagulation, molecular hydrogen is being formed inside the cloud.  The formation of molecular hydrogen can be summarized using either the two body reaction, three body reaction, formation using a free electron or proton, or surface formation \citep{krumholz2014}.  The two body formation scenario is the simplest reaction using two hydrogen atoms to produce molecular hydrogen, reaction \ref{eq:2bod}.

\begin{equation}\label{eq:2bod}
  \ce{H + H -> H_2}
\end{equation}

However the two body formation is not the major mechanism in the formation of H$_2$ due to the requirement of a forbidden photon that arises from the combination of hydrogen atoms in the ground state \citep{gould1963}.  If one of the hydrogen atoms is excited, the transition can occur and molecular hydrogen is formed, but the amount of excited hydrogen atoms in the temperature ranges typical of the cold and warm phases of the ISM are expected to be nearly nonexistent \citep{krumholz2014}.

The second formation scenario listed, the three body formation, involves three hydrogen atoms coming together to form molecular hydrogen with a spare hydrogen atom shown in reaction \ref{eq:3bod}.

\begin{equation}\label{eq:3bod}
  \ce{3H -> H_2 + H}
\end{equation}

The required density for the three body formation to occur is on the order of 10$^8$ cm$^{-3}$ \citep{palla1983,abel1997}, while the typical GMC density is on the order of 300 cm$^{-3}$.  The disparity between these densities eliminates any possibility of this being the primary mechanism to form molecular hydrogen. 

An alternative to the two or three body reactions uses either an electron or proton to ionize the hydrogen forming either H$^-$ or H$_2^+$ \citep{krumholz2014}.  The chemical reaction involving an electron is shown in reaction \ref{eq:eform} and reaction utilizing a proton is shown in reaction \ref{eq:pform}
\begin{equation}\label{eq:eform}
  \begin{split}
    \ce{H + \it{e} -> H^- + \it{h\nu}} \\
    \ce{H^- + \it{e} -> H_2 + \it{e}}
  \end{split}
\end{equation}

\begin{equation}\label{eq:pform}
  \begin{split}
    \ce{H + H^+ -> H^+_2 + \it{h\nu}} \\
    \ce{H^+_2 + H -> H_2 + H^+}
  \end{split}
\end{equation}

The main limitation of the free electron/proton formation mechanism is an undersupply of free electrons and protons.  Typical Milky Way conditions show that regions $>$1 cm$^{-3}$ have free electron and proton densities $<$10$^{-4}$ cm$^{-3}$ \citep{wolfire2003}.  Secondly, the H$^-$ ion in reaction \ref{eq:eform} is more likely to interact with a stray photon than an electron resulting in the ion returning to the atomic state \citep{glover2003}.  While we observe singly ionized hydrogen regions, HII regions, only a small fraction of the region will successfully produce molecular hydrogen.

While the free electron/proton formation method is suspected to be the primary H$_2$ catalyst in the early universe \citep{herbst2005}, at low redshifts the formation of molecular hydrogen on the surface of dust grains is the dominant mechanism of formation \citep{krumholz2014}.  The surface formation of molecular hydrogen will occur when a hydrogen atom strikes a dust grain and successfully sticks to the grain.  The hydrogen then will then interact with another hydrogen atom to form H$_2$ and be ejected from the dust particle after the reaction has occurred \citep{pirronello1997}.  The same process occurs on the dust grain as the two body formation, but  the dust grain will act as a medium to absorb the energy that would create the forbidden photon \citep{krumholz2014}.

%it would be a good idea to explain the requirements of the dust since not all dust can form H_2

With a dominant mechanism for molecular hydrogen formation, a reaction rate can be defined based on the cross section of the grain, $\Sigma_{gr}$, a striking probability dependent on the temperature of the observed dust, S(T), the probability of molecular hydrogen forming on the grain, $\upepsilon_{H_2}$, the density of free hydrogen in the GMC, n$_H$, and the density of hydrogen attached to the grain surface, n$_H{_0}$ \citep{krumholz2014}.  Scaling this with the integrated collisional probability of a Maxwellian gas, a reaction rate is determined to be equation \ref{eq:reac_rate} where k$_b$ is the Boltzmann constant, T is the temperature, and m$_H$ is the mass of the hydrogen atom.

\begin{equation}\label{eq:reac_rate}
  \frac{dn_{H_2}}{dt} = \frac{1}{2}\left(\frac{8k_bT}{\pi m_H}\right)^\frac{1}{2}\Sigma_{gr}S\left(T\right)\upepsilon_{H_2}n_{H}n_{H_0}
\end{equation}

Equation \ref{eq:reac_rate} is often simplified to equation \ref{eq:reac_rate_sm} by introducing a variable known as the formation rate, $\mathcal{R}_{gr}$, that is constrained using the column densities CI, CII, HI, H$_2$.  Typical reaction rates for the Milky Way have been found to be 3x10$^{-17}$cm$^3$s$^{-1}$ \citep{jura1975, gry2002, wolfire2008}.

\begin{equation}\label{eq:reac_rate_sm}
  \frac{dn_{H_2}}{dt} = \mathcal{R}_{gr}n_H n_{H_0}
\end{equation}

\subsection{Dissociation of Molecular Hydrogen}\label{h2destroy}

When the molecular hydrogen has formed, it is still susceptible to photodissociation from far ultra-violet (FUV) photons.  The energy required for a single photon to break the bonds of molecular hydrogen is 14.5 eV \citep{krumholz2014}.  Conveniently, this is also enough energy to excite atomic hydrogen, so photodissasociation using a single photon is highly unlikely due to the abundance of HI in the ISM \citep{krumholz2014}.  However, a photon with an energy of 11-13.6 eV will not be able to excite atomic hydrogen, but will be able to excite the molecular hydrogen to its first and second excitation levels, the Lyman and Werner bands respectively.  The excited H$_2$ will eventually settle via photon emission to its ground state with a finite probability of returning to two hydrogen atoms rather than maintaing its molecular state \citep{krumholz2014}.

%need a little more in this use the draine 2011 book
A dissociation rate, $\zeta_{diss}$, can be obtained by scaling the total excitation rate, $\zeta_{exc}$, by the fraction of excited hydrogen molecules that will settle to an atomic ground state \citep{krumholz2014}.  The total excitation rate is found by summing each of individual excitations from the ground state e.g. $\zeta_{exc,0-1}$, $\zeta_{exc,0-2}$.  In the Milky Way's diffuse ISM, the interstellar radiation field is 6-9x10$^{-14}$ erg cm$^{-3}$ over the range of 6-13.6 eV resulting in $\zeta_{exc} \approx 3\times10^{-10}$s$^{-1}$ \citep{draine2011}.  The expected fraction of H$_2$ to disassociate is between 0.11 and 0.13 resulting in $\zeta_{diss}\approx4\times10^{-11}$s$^{-1}$\citep{draine2011}.
Equating the formation and dissociation rates and then solving for a molecular hydrogen vs atomic hydrogen ratio gives in equation \ref{eq:fd_eq1} \citep{krumholz2014}.  From equation \ref{eq:fd_eq1} we can see, in the diffuse medium, atomic hydrogen is far more abundant than molecular hydrogen, as expected.

\begin{equation}\label{eq:fd_eq1}
  \begin{split}
    \zeta_{diss} n_{H_2} & = \mathcal{R}n_{H_0}n_H \\
    \frac{n_{H_2}}{n_H} & = \frac{\mathcal{R}n_{H_0}}{\zeta_{diss}} \\
                        & = 8\times10^{-6}\left(\frac{4\times10^{-11}s^{-1}}{\zeta_{diss}}\right)\left(\frac{n_{H_0}}{10cm^{-3}}\right)
  \end{split}
\end{equation}

However, as the density of the gas increases via collapse or coagulation, the optical depth will also increase limiting the amount of FUV photons able to penetrate into the core of the cloud, a process known as shielding \citep{draine2011}.  This shielding will aid in the decrease of $\zeta_{diss}$ which will increase the value in equation \ref{eq:fd_eq1}.  The increase in molecular to atomic hydrogen ratio signifies the accumulation of molecular gas reservoirs.  The molecular gas build up will lead to fragmentation within the cloud, and will eventually lead to forming stars.  

\section{Determining H$_2$ Abundance}

From the previous section, the formation of molecular hydrogen will result in the energy released during creation to be absorbed by a dust grain rather than through emission resulting in H$_2$ creation being a dark process, \S \ref{h2form}.  Furthermore, since the molecule is made of two hydrogen atoms, the high symmetry and low mass will negate any permanent dipole effects \citep{bolatto2013,kennicutt2012}.  This does not mean the H$_2$ is unexcitable, but the temperatures required to excite molecular hydrogen are (T $\gtrsim$ 100K)  above the temperature of a typical GMC\citep{bolatto2013}, and the settling of H$_2$ will more often than not result in the molecule separating into two hydrogen atoms \S \ref{h2destroy}.  For our purposes, we can consider molecular hydrogen a dark molecule requiring special treatment to determine the amount present.

Calculating the amount of molecular hydrogen in a system can be performed in several ways, and for the purpose of extragalactic sources they all involve using the amount of a molecular tracer to determine how much H$_2$ is present.  Molecules such as OH have been used in the past however mainly for Milky Way targets \citep{barrett1964}.  The molecule most commonly used in extragalactic studies is CO due to it's plentiful amount and ability to be easily observed \citep{bolatto2013}.  The amount of CO determined can then be converted to H$_2$ using a conversion factor (either X$_{CO}$ or $\alpha_{CO}$ depending on whether the final solution is in terms of column density or surface density).

%give some values and limitations for xco and aco here

\subsection{Methods for Determining CO-to-H$_2$ Conversion Factor}

Several methods are present to determine the conversion factor each with their own caveats.  One common method used to determine a conversion factor is though utilization of the virial nature of GMCs \citep{bolatto2013}.  This method works well for well defined clouds, however in the case of more distant nearby galaxies, the issue of whether or not the GMAs will display the same virialization as their constituent GMCs can hinder the results \citep{bolatto2013}.  The second caveat of using the virial mass is this method will only trace CO bright regions.  The exclusion of any possible CO-faint sections have resulted in a showing no dependence with the targets metalicity and the CO-to-H$_2$ conversion factor where a dependence has been noted using other methods \citep{bolatto2013}.  %maybe include something mentioning CO has a higher disassociation rate than H2?

A second method to ascertain a conversion factor is to incorporate observations of isotopologues of CO that are optically thin, commonly $^{13}$CO \citep{bolatto2013}.   The temperature, density, and column or surface density of the $^{13}$CO can be used to restrict the possible outcomes of the physical conditions of the GMC or GMA being examined.  This method shares the same problem as the virial technique in that it only probes CO bright regions of the target missing any CO-faint H$_2$\citep{bolatto2013}.  Another problem intrinsic to the use of optically thin tracers is a degeneracy in the models, such that multiple configurations can lead to the same observable outcomes \citep{bolatto2013}.

%\subsection{CO-to-H$_2$ Using Dust Emission}
The third major way to determine a CO-to-H$_2$ conversion factor is by incorporating the emission from dust to determine the amount of molecular gas present.  This assumes that the gas is well mixed with a constant ratio of dust and gas present in the galaxy \citep{leroy2011}, which has been shown to be true for the Milky Way \citep{boulanger1996}.  A suitable conversion factor is found by solving equation \ref{eq:aco_dgr}.

\begin{equation}\label{eq:aco_dgr}
  \begin{split}
    \delta_{GDR}\Sigma_{dust} & = \Sigma_{H_2} + \Sigma_{HI} \\
    						  & = \alpha_{CO} I_{CO} + \Sigma{HI}
  \end{split}
\end{equation}

Where $\Sigma_{dust}$, $\Sigma_{H_2}$, and $\Sigma_{HI}$ are the respective surface densities, I$_{CO}$ is the CO line intensity, $\alpha_{CO}$ is the conversion factor in M$_\odot$ pc$^{-2}$ K$^{-1}$ km$^{-1}$ s, and $\delta_{GDR}$ is the total mass of the gas divided by the total mass of the dust known as the dust-to-gas ratio \citep{leroy2011,sandstrom2013}.  In equation \ref{eq:aco_dgr}, we can measure $\Sigma_{dust}$, $\Sigma_{HI}$, and I$_{CO}$ leaving only the conversion factor and dust-to-gas ratio free to vary.  An appropriate $\alpha_{CO}$ value will correspond with a dust-to gas-ratio that is constant over the galaxy being studied, and has been carried out extensively by \cite{sandstrom2013} on kpc$^2$ scales.

Determining a conversion factor in this manner, allows the capability to trace the CO faint regions of target unlike the virial method or by using isotopologues of CO \citep{israel1997}.  Despite this advantage over the other two methods, using equation \ref{eq:aco_dgr} leaves any gas not associated with atomic hydrogen to be left as H$_2$ increasing the conversion factor and overall amount of molecular hydrogen \citep{bolatto2013}.  Another caution of this method rests with the assumption of a constant gas-to-dust ratio \citep{bolatto2013}.  Assuming a constant gas-to-dust ratio would imply a constant metalicity over the entire target galaxy due to the relation between the dust-to-gas ratio and metallicity shown in \cite{draine2007}.  Nevertheless, given the agreement of conversion factors between this method and other methods used, suggests that this added bit of gas, incorrectly assumed to be H$_2$, and any local fluctuations in metallicity of the target galaxy have very little effect on the final results, \citep{bolattor2013}.

%motivation (thronson)
%model (leroy and thronson)
%results (leroy 2009 and sandstrom)
%caveats (bolatto)




